%%%%%%%%%%%%%%%%%%%%%%%%%%%%%%%%%%%%%%%%%
% Medium Length Professional CV
% LaTeX Template
%
% This template has been downloaded from:
% http://www.LaTeXTemplates.com
%
% Original author:
% Trey Hunner (http://www.treyhunner.com/)
%
% Important note:
% This template requires the resume.cls file to be in the same directory as the
% .tex file. The resume.cls file provides the resume style used for structuring the
% document.
%
%%%%%%%%%%%%%%%%%%%%%%%%%%%%%%%%%%%%%%%%%

%----------------------------------------------------------------------------------------
%	PACKAGES AND OTHER DOCUMENT CONFIGURATIONS
%----------------------------------------------------------------------------------------

\documentclass{resume} % Use the custom resume.cls style

\usepackage[left=0.75in,top=0.6in,right=0.75in,bottom=0.6in]{geometry} % Document margins
\usepackage{verbatim}
\usepackage{color,hyperref}

\name{Vishwesh Jatala} % Your name
\address{Email: \href{mailto:vishwesh.jatala@austin.utexas.edu}{vishwesh.jatala@austin.utexas.edu}. Homepage: \href{https://vishweshjatala.github.io/}{https://vishweshjatala.github.io/}} % Your phone number and email

\address{\parbox{0.7\linewidth}{%
		\centering
        Office: 4.120, 201 E 24th Street, The University of Texas at Austin\\ %
        Austin, Texas, Zip Code: 78712, United States, Phone: +1 737-217-9808}
    }
%\address{\parbox{0.4\linewidth}}{Office: 4.120, 201 E 24th Street, The University of Texas at Austin\\, Austin, Texas, Zip Code: 78712, United States, Phone: +1 737-217-9808 } % Your address


\begin{document}

\begin{rSection}{Employment}
{\bf Postdoctoral Fellow } \hfill {\em May 2018 - Present} \\ 
Oden Institute for Computational Engineering and Sciences \\
The University of Texas at Austin \\
Supervisor: Prof. Keshav Pingali
\end{rSection}

\begin{rSection}{Research Interests}
Computer Architecture (GPU and CPU), Optimizing Compilers for GPU Performance and Energy, High Performance Computing, Source to Source Translators, Parallelization, and  GPU Graph Analytics.
\end{rSection}

%----------------------------------------------------------------------------------------
%	EDUCATION SECTION
%----------------------------------------------------------------------------------------

\begin{rSection}{Education}



{\bf Doctor of Philosophy} \hfill {\em July 2011 - Dec 2018} \\ 
Department of Computer Science \& Engineering \\
Indian Institute of Technology, Kanpur \\
CPI: 9.0/10 \\
Thesis Advisor: Prof. Amey Karkare


{\bf Bachelor of Technology} \hfill {\em July 2005 - May 2009}  \\
Department of Computer Science \& Engineering \\
Visvesvaraya National Institute of Technology, Nagpur \\
CPI: 9.23/10 (\textbf{Institute Medal})

%{\bf Nalanda Junior College, Gangur} \hfill {\em June 2002-04} \\ 
%Board of Intermediate Education, Andhra Pradesh \\
%Percentage: 95.4\%

\end{rSection}



\begin{rSection}{Publications}
\begin{itemize}
\item \textbf{Vishwesh Jatala}, Roshan Dathathri, Gurbinder Gill, Loc Hoang, V Krishna Nandivada, Keshav Pingali, A Study of Graph Analytics for Massive Datasets on Large-Scale Distributed GPUs, in 34th IEEE International Parallel \& Distributed Processing Symposium (\textbf{IPDPS}), 2020. %(\textit{To appear}).
\item Roshan Dathathri, Gurbinder Gill, Loc Hoang, Hoang-Vu Dang, \textbf{Vishwesh Jatala}, V Krishna Nandivada, Marc Snir, Keshav Pingali, Gluon-Async: A Bulk-Asynchronous System for Distributed and Heterogeneous Graph Analytics, in ACM/IEEE International Conference on Parallel Architectures and Compilation Techniques (\textbf{PACT}), 2019 \textbf{[Nominated for Best Paper]}.
\item Loc Hoang*, \textbf{Vishwesh Jatala*}, Xuhao Chen, Udit Agarwal, Roshan Dathathri, Gurbinder Gill, Keshav Pingali, DistTC: High Performance Distributed Triangle Counting, in IEEE High Performance extreme Computing Conference (\textbf{HPEC}), 2019.  [* Both authors contributed equally] \textbf{[Student Innovation Award]}.
\item \textbf{Vishwesh Jatala}, Jayvant Anantpur, and Amey Karkare, Reducing GPU Register File Energy, in 24th International European Conference on Parallel and Distributed Computing (\textbf{Euro-Par}), 2018. 
\item \textbf{Vishwesh Jatala}, Jayvant Anantpur, and Amey Karkare, GREENER: A Tool for Improving Energy Efficiency of GPU Register File, in High Performance Computing, Data, and Analytics, Student Research Symposium (\textbf{HiPC, SRS}), Jaipur, India, 2017 \textbf{[Best Poster Award]}. 
\item \textbf{Vishwesh Jatala}, Jayvant Anantpur, and Amey Karkare, Scratchpad Sharing in GPUs, in ACM Transactions on Architecture and Code Optimization (\textbf{TACO}), 2017. 
\item \textbf{Vishwesh Jatala}, Jayvant Anantpur, and Amey Karkare, Improving GPU Performance Through Resource Sharing, 25th Symposium on High-Performance Parallel and Distributed Computing (\textbf{HPDC}), Kyoto, Japan, 2016.
\item \textbf{Vishwesh Jatala}, Jayvant Anantpur, and Amey Karkare, Resource Sharing for GPUs, Code Generation and Optimization (\textbf{CGO}, Poster Track), Barcelona, Spain, 2016.




%\item \textit{Vishwesh Jatala}, Jayvant Anantpur, and Amey Karkare,  GREENER: A Tool for Improving Energy Efficiency of Register Files, CoRR, https://arxiv.org/abs/1709.04697, 2017. 
\end{itemize}
\end{rSection}


\begin{rSection}{Articles in Review/Submission}
\begin{itemize}
\item \textbf{Vishwesh Jatala}, Loc Hoang, Roshan Dathathri, Gurbinder Gill, V Krishna Nandivada, Keshav Pingali, An Adaptive Load Balancer For Graph Analytical Applications on GPUs, Arxiv, 2019. (\textbf{Under Submission})
\item Hochan Lee, Loc Hoang, \textbf{Vishwesh Jatala}, Roshan Dathathri, Gurbinder Gill, Keshav Pingali, A Study of Matrix-based APIs for Graph Analytics (\textbf{Under Review in PLDI 2020})

\end{itemize}
\end{rSection}

%\pagebreak


%----------------------------------------------------------------------------------------
%	Ph.D Thesis
%----------------------------------------------------------------------------------------

\begin{comment}
\begin{rSection}{Ph.D Thesis}{\hspace{6 mm}   \textit{Thesis Supervisor : Prof. Amey Karkare, Department of CSE, IIT Kanpur }}

\begin{itemize}
\item \textbf{Title:} Hardware and Software Optimizations for GPU Resource Management
%\textbf{Abstract:} Graphics Processing Units (GPUs) provide lot of opportunities for parallelization. Recently, they have been used for general purpose computations in addition to graphic computations. And GPUs can provide large benefits when they are used for data processing frameworks/languages such as MapReduce and SQL since they possess lot of parallelism. However, manual parallelization involves an overhead in optimizing for GPU architecture. To reduce this, we propose an auto parallelization framework for MapReduce which can be efficiently executed on GPU.
\end{itemize}

\end{rSection}
\end{comment}

%----------------------------------------------------------------------------------------
%	ACADEMIC ACHIEVEMENTS
%----------------------------------------------------------------------------------------


\begin{rSection}{Academic Achievements}
\begin{itemize}
\item Recipient of \textbf{Student Innovation Award} in HPEC, 2019
\item \textbf{Nominated for Best Paper Award} in PACT, 2019
\item Recipient of \textbf{Tata Consultancy Services (TCS) Ph.D. Fellowship} from Jan 2014 to June 2018.
\item Recipient of \textbf{Best Poster Award} in HiPC, Student Research Symposium, 2017
\item Recipient of \textbf{Best Poster Award} in IBM Research Day, IIT Kanpur, 2017
\item Recipient of \textbf{Institute Medal} for academic excellence in B.Tech, CSE, VNIT Nagpur, 2009.
\item Awarded \textbf{Academic Excellence Prize} for academic excellence in B.Tech, CSE, VNIT Nagpur, 2009.
\item Recipient of \textbf{Dr.V.M. Dokras Felicitation Committee Prize} for academic excellence in $3^{rd}$ year B.Tech, CSE, VNIT Nagpur, 2008.
\item Recipient of \textbf{Academic Excellence Prize} for academic excellence in $3^{rd}$ year B.Tech, CSE, VNIT Nagpur, 2008.
\item Recipient of \textbf{Dr.S.G.Ghangrekhar Prize} for excellence in Mathematics in B.Tech, VNIT Nagpur, 2006
\end{itemize}
\end{rSection}



%----------------------------------------------------------------------------------------
%	WORK EXPERIENCE SECTION
%----------------------------------------------------------------------------------------

\begin{rSection}{Past Work Experience}
\begin{itemize}

\item \begin{rSubsection}{IBM India Research Laboratory}{\textit{May 2013 - July 2013}}{Research Intern}{New Delhi}
\textit{Project Title}: Reliable Multicast using Software Defined Networking 
%In this project, we provided reliability service for group communication. The project extends Avalanche multi-cast routing algorithm by supporting reliability using pragmatic general multicast (PGM) protocol.
\end{rSubsection}


%\item \begin{rSubsection}{IIT Kanpur}{\textit{Aug 2011 - July 2012}}{System Administrator}{Kanpur}
%The project duties involve in resolving departmental network and computer laboratory issues. 
%\end{rSubsection}


\item \begin{rSubsection}{Oracle India Pvt Ltd}{\textit{June 2009 - July 2011}}{Member of Technical Staff}{Hyderabad}
The aim of our project is to find the security vulnerabilities in a product and provide best possible solutions to make it secure.
\end{rSubsection}


\end{itemize}
\end{rSection}



\begin{rSection}{Talks/Presentations}
\begin{itemize}
\item \textit{Scratchpad Sharing in GPUs}, at 12th Inter-Research-Institute Student Seminar in Computer Science (IRISS), Nagpur, Feb 2018. 
\item \textit{Scratchpad Sharing in GPUs}, CSE Doctoral Symposium, NIIT University, Rajasthan, September 23rd-24th, 2017.
\item Poster Presentation on \textit{Resource Sharing for GPUs}, IBM Research Day, IIT Kanpur, April 2017. \textbf{[IBM Best Poster Award]} 
\item \textit{Improving GPU Performance Through Resource Sharing}, 11th Inter-Research-Institute Student Seminar in Computer Science (\textbf{IRISS}), Kolkata, Jan 2017.
\item Poster Presentation on \textit{Resource Sharing for GPUs}, Technology Day, IIT Kanpur, May 016. 

\end{itemize}
\end{rSection}

%----------------------------------------------------------------------------------------
%	TECHNICAL STRENGTHS SECTION
%----------------------------------------------------------------------------------------

\begin{rSection}{Technical Skills}

\begin{tabular}{ @{} >{\bfseries}l @{\hspace{6ex}} l }
Open Source Simulators/Tools & GPGPU-Sim, GPU Ocelot, Cetus, GPUWattch \\
& CACTI, McPAT, Soot, Lex, Yacc, PIN \\
Programming Languages & C, C++, Java, and Pascal \\
Assembly Level Languages & PTX and MIPS\\
Scripting Languages &  Perl Script and Shell Script\\
Parallel Programming & CUDA, OpenMP, and MPI \\
Network	Programming & RPC, RMI, and  Sockets \\
Query Languages & SQL, LINQ, and SPARQL\\
%Web Designing & HTML and PHP\\
\end{tabular}

\end{rSection}

\begin{rSection}{Teaching Assistant}
%Designing course assignments, mentoring students for course projects, grading. \\
\begin{tabular}{ll}
CS738: Advanced Compiler Optimizations & CS335: Compiler Design \\ 
CS601: Mathematics for Computer Science & CS220: Introduction to Computer Organization \\  
CS252: Computing Laboratory & CS639: Program Analysis, Verification and Testing\\
NPTEL: Fundamentals of Database System &   CS602: Design and Analysis of Algorithms  \\
Workshop on C Programming \& Data Structures 
\end{tabular}

\end{rSection}


%----------------------------------------------------------------------------------------
%	Professional Service
%----------------------------------------------------------------------------------------


\begin{rSection}{Professional Service}

\begin{itemize}
\item Reviewer for IPDPS 2020, PACT 2019, HIPC SRS 2017, ICCI 2017, and TOPC 2016
\item Member of PPoPP 2020 Artifact Evaluation Committee
\item Evaluated master thesis of a student at NTNU: Norwegian University of Science and Technology university
\end{itemize}

\end{rSection}


%----------------------------------------------------------------------------------------
%	References
%----------------------------------------------------------------------------------------


\begin{rSection}{References}

\begin{itemize}
\item Prof. Amey Karkare, Associate Professor, Department of CSE, IIT Kanpur, Kanpur, India. \\
Email: karkare@cse.iitk.ac.in, Phone: +91 512 259 7520
\item Prof. Keshav Pingali, Professor, Department of CS, The University of Texas at Austin, USA. \\
Email: pingali@cs.utexas.edu, Phone: +1 512 232 6567
%\item Prof. Mainak Chaudhuri, Associate Professor, Department of CSE, IIT Kanpur, Kanpur, India. \\
%Email: mainakc@cse.iitk.ac.in, Phone: +91 512 259 7890
\item Dr. Jayvant Anantpur, Staff Engineer, Mentor Graphics Inc, Portland, USA. \\
Email: jayvant.anantpur@gmail.com, Phone: +1 503 410 6637
\end{itemize}

\end{rSection}

\pagebreak



\begin{comment}

%----------------------------------------------------------------------------------------
%	ACADEMIC PROJECTS
%----------------------------------------------------------------------------------------

\begin{rSection}{Academic Course Projects}

\begin{itemize}
\item \textbf{B.Tech Project: Automatic Number Plate Detection System using HTM}  \\
%\textbf{Description:} 
In this project, we implemented a number plate recognition system which can extract registration number from the image of a vehicle that is located at a suitable distance from a camera. We used image processing techniques to extract the number plate from the image, and recognized the characters in the number plate using hierarchical temporal memory (HTM).
\item \textbf{CS622: Cache Replacement Policy using Future and Reuse Distance}  \\
%\textbf{Description:}
LRU cache replacement policy fails to minimize cache misses when applications do not support temporal locality. To minimize the cache misses, we propose an approach that emulates the optimal replacement policy by introducing future and reuse distance in the cache.
\item  \textbf{CS738: Implementing Higher Level Loop Optimizations} \\
%\textbf{Description:}
We implemented the following loop optimizations for parallelizing loops in a program: loop interchange, loop skewing, loop reversal, and loop selection.
\item \textbf{CS618: Semantic Searching Technique in Temporal Database} \\
%\textbf{Description:}
Semantic searching techniques yield better results than the syntactic searching techniques. In this project, we implemented a technique to semantically search for given text from a temporal database. The proposed implementation uses RDF graph to represent the temporal information and uses R* tree to index the given the data.
\item \textbf{CS632: Distributed Storage System for Mobile Devices} \\
%\textbf{Description:}
In this project, I proposed an approach for distributed storage system in mobile devices and retrieval of data from the system even in the presence of failure of nodes. The proof of concept is verified by developing a utility, Distributed Phone Book. 
\end{itemize}
\end{rSection}


%----------------------------------------------------------------------------------------
%	EXTRA CURICULAR ACTIVITIES
%----------------------------------------------------------------------------------------

\begin{rSection}{Extra Curricular Activities}

\begin{itemize}
\item Organizing member for the event of \textit{Paper Presentation} in AXIS-2006 at VNIT Nagpur
\item Participated in the event of \textit{Contraption} in AXIS-2006 at VNIT Nagpur
\item Participated in gaming event of \textit{IBM Software Context} in AXIS-2006 at VNIT Nagpur
\item Finalist in the event of \textit{Algo-Rhythm} in AXIS-2008  at VNIT Nagpur
\end{itemize}
\end{rSection}

\end{comment}





\end{document}
